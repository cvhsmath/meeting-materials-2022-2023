\documentclass[11pt,letterpaper]{article}
\usepackage[lmargin=1in,rmargin=1in,tmargin=1in,bmargin=1in]{geometry
\graphicspath{ {./images/}
}


% -------------------
% Packages
% -------------------
\usepackage{
	amsmath,			% Math Environments
	amssymb,			% Extended Symbols
	enumerate,		    % Enumerate Environments
	graphicx,			% Include Images    
	lastpage,			% Reference Lastpage
	multicol,			% Use Multi-columns
	multirow			% Use Multi-rows
}


% -------------------
% Font
% -------------------
\usepackage[T1]{fontenc}
\usepackage{charter}    


% -------------------
% Heading Commands
% -------------------
\newcommand{\class}{Mu Alpha Theta}
\newcommand{\term}{2022-2023}
\newcommand{\head}[2]{%
\thispagestyle{empty}
\vspace*{-0.5in}
\noindent\begin{tabular*}{\textwidth}{l @{\extracolsep{\fill}} r @{\extracolsep{6pt}} l}
	\textbf{#1} & \textbf{Name:} & \makebox[5.75cm]{\hrulefill} \\
	\textbf{#2} & & \\
	\textbf{\class:\; \term} & & \\
\end{tabular*} \\
\rule[2ex]{\textwidth}{2pt} %
}


% -------------------
% Commands
% -------------------
\newcommand{\prob}{\noindent\textbf{Problem. }}
\newcounter{problem}
\newcommand{\problem}{
	\stepcounter{problem}%
	\noindent \textbf{Problem \theproblem. }%
}
\newcommand{\pointproblem}[1]{
	\stepcounter{problem}%
	\noindent \textbf{Problem \theproblem.} (#1 points)\,%
}
\newcommand{\pspace}{\par\vspace{\baselineskip}}
\newcommand{\ds}{\displaystyle}


% -------------------
% Header & Footer
% -------------------
\usepackage{fancyhdr}

\fancypagestyle{pages}{
	%Headers
	\fancyhead[L]{}
	\fancyhead[C]{}
	\fancyhead[R]{}
\renewcommand{\headrulewidth}{0pt}
	%Footers
	\fancyfoot[L]{}
	\fancyfoot[C]{}
	\fancyfoot[R]{}
\renewcommand{\footrulewidth}{0.0pt}
}
\headheight=0pt
\footskip=14pt

\pagestyle{pages}


% -------------------
% Content
% -------------------and
\begin{document}
\head{Worksheet \#}{Date:}


% Question 1: https://artofproblemsolving.com/wiki/index.php/2014_AMC_8_Problems/Problem_9
\problem \pspace
\\
\begin{center} \includegraphics[scale=0.2]{images/AMC-8-P9.png} \end{center}
\\ \pspace
\noindent In $\bigtriangleup ABC$, $D$ is a point on side $\overline{AC}$ such that $BD=DC$ and $\angle BCD$ measures $70^\circ$. What is the degree measure of $\angle ADB$? \vspace{2cm}


% Question 2: https://artofproblemsolving.com/wiki/index.php/2014_AMC_8_Problems/Problem_14
\problem \pspace
\\
\begin{center} \includegraphics[scale=0.25]{images/AMC-8-P14.png} \end{center} 
\\ \pspace
\noindent Rectangle $ABCD$ and right triangle $DCE$ have the same area. They are joined to form a trapezoid, as shown. What is $DE$?
\vspace{2.25cm}


% Question 3: https://artofproblemsolving.com/wiki/index.php/2014_AMC_8_Problems/Problem_14
\problem \pspace
\\
\noindent A cube with $3$-inch edges is to be constructed from $27$ smaller cubes with $1$-inch edges. Twenty-one of the cubes are colored red and $6$ are colored white. If the $3$-inch cube is constructed to have the smallest possible white surface area showing, what fraction of the surface area is white?
\pagebreak


% Question 4: https://artofproblemsolving.com/wiki/index.php/2014_AMC_10B_Problems/Problem_5
\problem \pspace
\\
\begin{center} \includegraphics[scale=0.25]{images/AMC-10-P4.png} \end{center} 
\\ \pspace
\noindent Doug constructs a square window using $8$ equal-size panes of glass, as shown. The ratio of the height to width for each pane is $5 : 2$, and the borders around and between the panes are $2$ inches wide. In inches, what is the side length of the square window?
\vspace{4cm}


% Question 5: https://artofproblemsolving.com/wiki/index.php/2014_AMC_10B_Problems/Problem_13
\problem \pspace
\\
\begin{center} \includegraphics[scale=0.25]{images/AMC-10-P13.png} \end{center} 
\\ \pspace
\noindent Doug constructs a square window using $8$ equal-size panes of glass, as shown. The ratio of the height to width for each pane is $5 : 2$, and the borders around and between the panes are $2$ inches wide. In inches, what is the side length of the square window?
\vspace{4cm}

\end{document}			